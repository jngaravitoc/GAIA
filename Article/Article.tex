% mn2esample.tex
%
% v2.1 released 22nd May 2002 (G. Hutton)
%
% The mnsample.tex file has been amended to highlight
% the proper use of LaTeX2e code with the class file
% and using natbib cross-referencing. These changes
% do not reflect the original paper by A. V. Raveendran.
%
% Previous versions of this sample document were
% compatible with the LaTeX 2.09 style file mn.sty
% v1.2 released 5th September 1994 (M. Reed)
% v1.1 released 18th July 1994
% v1.0 released 28th January 1994

\documentclass[useAMS,usenatbib]{mn2e}
\usepackage{graphicx}
\usepackage{natbib}
\usepackage[a4paper,breaklinks,dvipdfm]{hyperref}

% If your system does not have the AMS fonts version 2.0 installed, then
% remove the useAMS option.
%
% useAMS allows you to obtain upright Greek characters.
% e.g. \umu, \upi etc.  See the section on "Upright Greek characters" in
% this guide for further information.
%
% If you are using AMS 2.0 fonts, bold math letters/symbols are available
% at a larger range of sizes for NFSS release 1 and 2 (using \boldmath or
% preferably \bmath).
%
% The usenatbib command allows the use of Patrick Daly's natbib.sty for
% cross-referencing.
%
% If you wish to typeset the paper in Times font (if you do not have the
% PostScript Type 1 Computer Modern fonts you will need to do this to get
% smoother fonts in a PDF file) then uncomment the next line
% \usepackage{Times}

%%%%% AUTHORS - PLACE YOUR OWN MACROS HERE %%%%%


%%%%%%%%%%%%%%%%%%%%%%%%%%%%%%%%%%%%%%%%%%%%%%%%

\title[]{GAIAS is view on the metallicity of the Milky Way Galaxy}
\author[J. G. Fern\'andez-Trincado and J. N. Garavito-Camargo]{
J. G. Fern\'andez-Trincado,$^{1,3}$\thanks{Email: jfernandez@cida.ve}
J. N. Garavito-Camargo,$^{2}$\thanks{Email: jngaravitoc@unal.edu.co}
F. Figueras,$^{4}$
K. Daisuke,$^{5}$ 
\newauthor
J. T. Mu\~noz,$^{6}$   
P. Marin$^{7}$
\\
$^{1}$Institute Utinam, CNRS UMR6213, Universit\'e de Franche-Comt\'e, OSU THETA de Franche-Comt\'e-Bourgogne, Besan\c{c}on, France\\
$^{2}$Departamento de F\'{i}sica, Universidad de los Andes, Cra. 1 No. 18A-10, Edificio Ip, Bogot\'a, Colombia\\
$^{3}$Centro de Investigaciones de Astronom\'ia, AP 264, M\'erida 5101-A, Venezuela\\
$^{4}$Escribi afiliacion Francesca\\
$^{5}$Escribi afiliacion Daisuke\\
$^{6}$Escribi afiliacion Jessica\\
$^{7}$Escribi afiliacion Patricia\\
} 

\begin{document}

\date{01 January 2014}

\pagerange{\pageref{firstpage}--\pageref{lastpage}} \pubyear{2002}

\maketitle

\label{firstpage}

\begin{abstract}

We presented the predictions of GAIA simulator and Hidrodynamical simulations
for red clump stars in the Milky Way in order to explore the metallicity gradients in the vertical and radial scale on the 
galactic disk.
\end{abstract}

\begin{keywords}
quasars: point objects -- surveys -- variability.
\end{keywords}

\section{Introduction}


\section{The Data}

\subsection{GOG: }

\subsection{Test particle simulation}

\section{SPH simulation}

We also use the SPH simulation made by (Daisuke et al. ) in 
which a Milky Way galaxy. 

\subsection{Simulation parameters}


This simulation have $10^3$ particles... Initial conditions etc.. Extinction model 


\section{Results and Conclussions}

We assume that all stars in the simulation are Red Clump
stars with $M_v=1.274$. Gaia is field of view would be constrained 
to $v<16.5$ (check this it could be larger), this field is illustrated in Fig. ??





\section*{Acknowledgments}
 
 
 
 Agradecer a GAIA. 



\begin{thebibliography}{99}


\bibitem[\protect\citeauthoryear{Baltay et al.}{2002}]{b14} Baltay, C. et al. 2002,
PASP, 114, 780

\bibitem[\protect\citeauthoryear{Brice\~no et al. 2005}{}]{b7} Brice\~no, C. et al. 2005,
AJ, 907, 926 

\bibitem[\protect\citeauthoryear{Brice\~no et al. 2011}{}]{b8} Brice\~no, C. et al. 2011,
RMxAC, 225, 226

\bibitem[\protect\citeauthoryear{Bongiovanni et al. 2005}{}]{b13} Bongiovanni, A. et al. 2005,
MNRAS, 930, 940

\bibitem[\protect\citeauthoryear{Calvet et al. 2005}{}]{b9} Calvet, N. et al. 2005,
AJ, 935, 946

\bibitem[\protect\citeauthoryear{Downes et al. 2009}{}]{b2} Downes, J. J. et al. 2009,
RevMexAA, 60, 61

\bibitem[\protect\citeauthoryear{Downes et al. 2008}{}]{b11} Downes, J. J. et al. 2008,
AJ, 51, 66

\bibitem[\protect\citeauthoryear{Downes et al.}{2012}]{b15} Downes, J. J. et al. 2002,
in preparation

\bibitem[\protect\citeauthoryear{Hern\'andez et al. 2009}{}]{b10} Hern\'andez, J. et al. 2009,
RMxAC, 68, 69

\bibitem[\protect\citeauthoryear{Mateu et al. 2009}{}]{b5} Mateu, C. et al. 2009,
ApJ, 4412, 4423 

\bibitem[\protect\citeauthoryear{Mateu et al.}{2012}]{b6} Mateu, C. et al. 2012,
MNRAS, 4599, 4623 

\bibitem[\protect\citeauthoryear{Rengstorf et al.}{2006}]{b1} Rengstorf, A.W. et al. 2006,
AJ, 1923, 1933

\bibitem[\protect\citeauthoryear{Schmidt et al.}{2010}]{b12} Schmidt, K. B. et al. 2010,
ApJ, 1194, 1208

\bibitem[\protect\citeauthoryear{Schlegel, Finkbeiner \& Davis}{1998}]{b16} Schlegel, D. J., Finkbeiner D. P.,
Davis M., 1998, ApJ, 500, 525.

\bibitem[\protect\citeauthoryear{Vivas et al.}{2004}]{b3} Vivas, A. K. et al. 2004,
AJ, 1158, 1175

\bibitem[\protect\citeauthoryear{Vivas \& Zinn 2006}{}]{b4} Vivas, A. K. \& Zinn, R., 2006,
AJ, 714, 728

\label{lastpage}

\end{document}



%\begin{table*}
% \centering
% \begin{minipage}{140mm}
%  \caption{Data on the RV Tauri stars detected by {\it IRAS}.}
%  \begin{tabular}{@{}llrrrrlrlr@{}}
%  \hline
%   Name     &            & \multicolumn{4}{c}{Flux density (Jy)%
%  \footnote{Observed by {\em IRAS}.}}\\
%   Variable & {\it IRAS} & 12$\,\umu$m & 25$\,\umu$m & 60$\,\umu$m
%     & 100$\,\umu$m & Sp. & Period & Light- & $T_0\,(\rmn{K})$ \\
%        &  &  &  &  &  & group & (d) & curve \\
%        &  &  &  &  &  &       &     & type  \\
% \hline
% TW Cam & 04166$+$5719 & 8.27 & 5.62 & 1.82 & $<$1.73 & A & 85.6 & a & 555 \\
% RV Tau & 04440$+$2605 & 22.53 & 18.08 & 6.40 & 2.52 & A & 78.9 & b & 460 \\
% DY Ori & 06034$+$1354 & 12.44 & 14.93 & 4.12 & $<$11.22 & B & 60.3 &  & 295 \\
% CT Ori & 06072$+$0953 & 6.16 & 5.57 & 1.22 & $<$1.54 & B & 135.6 &  & 330 \\
% SU Gem & 06108$+$2734 & 7.90 & 5.69 & 2.16 & $<$11.66 & A & 50.1 & b & 575 \\
% UY CMa & 06160$-$1701 & 3.51 & 2.48 & 0.57 & $<$1.00 & B & 113.9 & a & 420 \\
% U Mon  & 07284$-$0940 & 124.30 & 88.43 & 26.28 & 9.24 & A & 92.3 & b & 480 \\
% AR Pup & 08011$-$3627 & 131.33 & 94.32 & 25.81 & 11.65 & B & 75.0 & b & 450 \\
% IW Car & 09256$-$6324 & 101/06 & 96.24 & 34.19 & 13.07 & B & 67.5 & b & 395 \\
% GK Car & 11118$-$5726 & 2.87 & 2.48 & 0.78 & $<$12.13 & B & 55.6 &  & 405 \\
% RU Cen & 12067$-$4508 & 5.36 & 11.02 & 5.57 & 2.01 & B & 64.7 &  & 255 \\
% SX Cen & 12185$-$4856 & 5.95 & 3.62 & 1.09 & $<$1.50 & B & 32.9 & b & 590 \\
% AI Sco & 17530$-$3348 & 17.68 & 11.46 & 2.88 & $<$45.62 & A & 71.0 & b & 480 \\
% AC Her & 18281$+$2149 & 41.47 & 65.33 & 21.12 & 7.79 & B & 75.5 & a & 260 \\
% R Sct  & 18448$-$0545 & 20.88 & 9.30 & 8.10 & $<$138.78 & A & 140.2 & a \\
% R Sge  & 20117$+$1634 & 10.63 & 7.57 & 2.10 & $<$1.66 & A & 70.6 & b & 455 \\
% V Vul  & 20343$+$2625 & 12.39 & 5.72 & 1.29 & $<$6.96 & A & 75.7 & a & 690\\
%\hline
%\end{tabular}
%\end{minipage}
%\end{table*}


